\documentclass{beamer}

% Imported Packages
\usepackage[utf8]{inputenc}

% Theme
\usetheme[white, compactlogo]{Wisconsin}

%This block of code defines the information to appear in the
%Title page
\title{QT-Opt: Scalable Deep Reinforcement Learning for Vision-Based Robotic Manipulation}
\subtitle{arXiv:1806.10293, Kalashnikov et al, 2018.}
\author{\textit{Sumamrized by} Hyecheol (Jerry) Jang}
\institute{
  Department of Computer Sciences\\
  University of Wisconsin–Madison
}
\date{RL Paper Study, Jun. 29. 2020}
%End of title page configuration block
%------------------------------------------------------------


%------------------------------------------------------------
%The next block of commands puts the table of contents at the 
%beginning of each section and highlights the current section:

\AtBeginSection[]
{
  \begin{frame}
    \frametitle{Table of Contents}
    \tableofcontents[currentsection]
  \end{frame}
}
%------------------------------------------------------------


%---------------------------------------------------------
%This block defines the existing sections
\newcommand{\firstSec}{Motivation}
\newcommand{\secondSec}{Second Section}
% \newcommand{\nthSec}[n]{nth Section}
%---------------------------------------------------------


\begin{document}
  %The next statement creates the title page.
  \frame{\titlepage}

  % Section 1, Motivation
  \section{\firstSec}
    \begin{frame}
      \frametitle{\firstSec : Why Robotics + Reinforcement Learning}
      \begin{itemize}
        \item Usually, Robots are good at \textbf{repetitive tasks} (e.g. Assembly Line)
              \pause
        \item Want to make Robots that \textbf{identifies surroundings} and \textbf{behave accordingly},
              but it is difficult
              \pause
        \begin{itemize}
          \item \textbf{Deep Learning}\\
                Provide ability to handling real-world scenarios
          \item \textbf{Reinforcement Learning}\\
                Provide ability to make decision in long-term,
                using previous experiences in complex and robust scenarios
        \end{itemize}
        \pause
        \item Combining two techniques
        \begin{itemize}
          \item Able to learn policy continuously from their experience
          \item No need for manual engineering, use data they collects
        \end{itemize}
      \end{itemize}
    \end{frame}

    \begin{frame}
      \frametitle{\firstSec : Difficulites of Using RL in Robotics}
      \begin{itemize}
        \item Varience in \textbf{visual and physical property of objects}
        \pause
        \begin{itemize}
          \item Hardness of object (Soft or Hard)
          \item Surface Characteristics (Slippery, Sticky, \ldots)
          \item Color Variation
          \item Shape Variation
          \item \ldots
        \end{itemize}
        \pause
        \item \textbf{Noise} of sensors
      \end{itemize}
    \end{frame}

    \begin{frame}
      \frametitle{\firstSec : Previous Works}
    \end{frame}
  % End of Section 1


\end{document}
